\documentclass[11pt,a4paper,twoside,BCOR=1cm,DIV=11,headsepline]{scrreprt} %draft

%***************Start Paket-Einbindungen******************
\usepackage[latin1]{inputenc} %textcodierung, in windows auch latin1, in unix utf8
\usepackage[ngerman]{babel}		%deutsche Sprachtrennung
\usepackage{csquotes}
\usepackage{amsfonts,amsmath,amsthm}	
\usepackage{makeidx}	%fuer Verwendung des index'
\usepackage[colorlinks=false,pdfborder={0 0 0},plainpages=false]{hyperref} %fuer hyperlinks im pdf-format

%GENAU DANN verwenden, wenn digital veroeffentlicht
%\usepackage{sidecap} % um \caption{?} neben bildern m�glich zu machen
\usepackage{pgf} %fuer grafiken
\usepackage{here}
\usepackage{graphicx}

% BibLaTeX
%
% We are using the 'alphabetic' style.
% The default is the 'numerical' style.
%\usepackage[]{biblatex}
\usepackage[backend=biber]{biblatex}
%\usepackage[style=numeric,backend=biber]{biblatex}
%\usepackage[style=authoryear,backend=biber]{biblatex}
%\usepackage[style=alphabetic,backend=biber]{biblatex}
%\usepackage[style=alphabetic,backend=biber]{biblatex}
%
% BibLaTex Database
\addbibresource{literatur.bib}
%\bibliography{literatur}





% Glossar
%\usepackage{glossaries}
%***************Ende Paket-Einbindungen******************

%**************Start Allgemeine Optionen*****************
\setkomafont{sectioning}{\rmfamily\bfseries\boldmath}
\makeindex
%\hyphenation{geo-d�-tisch Geo-d�-te L�n-gen-raum un-ter-halb-ste-tig un-ter-halb-ste-ti-ge Ver-gleichs-drei-ecke Alex-an-drov} %hier geh�ren zu trennende W�rter hin, die Latex nicht kennt
%% Gleichungen nach Kapitel nummerieren
\renewcommand{\theequation}{\thechapter.\arabic{equation}}
\numberwithin{equation}{chapter}
%\setlength{\parindent}{0pt} %kein Einzug (nie)
%\addtolength{\leftmargini}{2.5em}
%****************Ende Allgemeine Optionen*************

%************** Start Befehle *************
%Aenderung der Nummerierungsart (klein roemisch)
\renewcommand{\labelenumi}{(\roman{enumi})}
%************** Ende Befehle **************


%************** Start Grafik **********
\providecommand{\graphic}[1]{\begin{center}{\footnotesize \input{graphics/#1.TpX}}\end{center}}
%************** Ende Grafik ***********