\chapter{Fazit}

In dieser Arbeit haben wir uns mit verschiedenen Fragestellungen befasst und nach neuen L�sungsans�tzen gesucht um auf einfache und effiziente Weise Inhalte f�r eine Mobiel App bereitzustellen und zu �bermitteln.

Wir hatten grosse Schwierigkeiten herauszufinden, welche Technologien und Verfahren bereits eingesetzt werden, da wir kein Produkt gefunden haben, welches sich prim�r mit der effizienten �bermittlung von Inhalten auf eine Mobile App befasst. Zus�tzlich sind fast keine technischen Informationen verf�gbar von Dienstleistern welche sich auf Mobile Apps spezialisiert haben. Das hat die Recherchearbeit erschwert und stellenweise auch etwas langatmig gemacht.

Trotzdem war die Aufgabe sehr spannend. Neben der Informationsbeschaffung und der Auseinandersetzung mit verschiedenen Technologien, haben wir ein eigenes CMS aufbauen k�nnen und eine App entwickelt, welche sich �ber dieses CMS aktualisieren l�sst. Das CMS und die App sind so aufgebaut, dass sich leicht neue Inhaltstypen und weitere Funktionalit�t implementieren l�sst.

Selbstverst�ndlich gibt es f�r die Zukunft noch viel Erweiterungspotential um welches wir uns im Rahmen dieser Arbeit nicht k�mmern konnten. Das System ist zwar so aufgebaut, dass man das CMS f�r mehrere Mandanten einsetzen k�nnte, diese Funktionalit�t ist aber noch nicht implementiert. Ebenso fehlt noch eine Userverwaltung und andere Sicherheitsaspekte wurden auch noch nicht ber�cksichtigt. ausserdem ist das L�schen von nicht mehr ben�tigten Bildern und anderen Assets auf dem Smartphone ein weiterer wichtiger Punkt.




