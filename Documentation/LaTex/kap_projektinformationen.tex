\chapter{Projektinformationen}


\section{Ausgangslage}

F�r content-basierte Mobile Apps m�ssen immer wieder neue Inhalte publiziert werden. Dies soll jeweils vom Betreiber der App auf
einfache Art m�glich sein. Dies ist m�glich, wenn die Inhalte in einem CMS verwaltet werden. Die App soll automatisch aktuell gehalten
werden, ohne dass die ganze App aktualisiert werden muss, und auch ohne dass der Konsument aktiv wird. Wichtig ist zudem, dass die
Daten auch offline zur Verf�gung stehen und auch gr�ssere Datenmengen unproblematisch �bermittelt werden k�nnen. Trotzdem
muss aber ein Mechanismus gefunden werden, dass nur m�glichst kleine Datenmengen �bermittelt werden m�ssen um z.B. auf
geringe Bandbreite reagieren zu k�nnen, oder Roaming-Geb�hren klein zu halten. Zum Beispiel m�ssen Apps von Museen, Ausstellungen, Konzertbetreibern oder Zeitschriften diese Anforderungen erf�llen

\section{Aufgabenstellung / Abgrenzung}


\section{Ziele der Arbeit}

Das Ziel der Seminararbeit ist es aufzuzeigen, welche Systeme, Prozesse und Design-Patterns wichtig sind und in Frage kommen um den oben genannten Anforderungen gerecht zu werden. Insbesondere die Schwierigkeit, m�glichst effizient aktuelle Daten dem App-Benutzer zur Verf�gung zu stellen soll dabei beachtet werden.

Falls m�glich k�nnen bereits gewisse Teile praktisch umgesetzt werden: z.B. kann ein CMS bereitgestellt werden, �ber welches der Betreiber der App die Inhalte bereitstellen und zur Aktualisierung f�r die App freigeben kann. Weiter kann eine App entwickelt werden, welche ihre Inhalte �ber das CMS bezieht und die definier 

\section{Motivation}
W�re doch nur die Motivation da...
...
...


\section{Projektverlauf}

blablabla...
....
...


\begin{itemize}
\item Punkt1
\item ...

\end{itemize}


\section{Hilfsmittel}

Um m�glichst effizient programmieren zu k�nnen, haben wir als Entwicklungsumgebung Eclipse \cite{Eclipse} gew�hlt.
\newline
\\Um eine \emph{Versionskontrolle} zu haben, und damit wir beide gleichzeitig am Projekt arbeiten konnten, haben wir bei Github \cite{Github} ein Repository erstellt und s�mtliche f�r das Projekt n�tigen Dateien dort eingecheckt.
\newline
\\Die Dokumentation haben wir mit \LaTeX{} erstellt.

Pencil Project \cite{pencilProject}
