\chapter{Einleitung}


\section{Ausgangslage}
F�r content-basierte Mobile Apps m�ssen immer wieder neue Inhalte publiziert werden. Dies soll jeweils vom Betreiber der App auf einfache Art m�glich sein. Dies ist m�glich, wenn die Inhalte in einem CMS verwaltet werden. Die App soll automatisch aktuell gehalten werden, ohne dass die ganze App aktualisiert werden muss, und auch ohne dass der Konsument aktiv wird. Wichtig ist zudem, dass die Daten auch offline zur Verf�gung stehen und auch gr�ssere Datenmengen unproblematisch �bermittelt werden k�nnen. Trotzdem muss aber ein Mechanismus gefunden werden, dass nur m�glichst kleine Datenmengen �bermittelt werden m�ssen um z.B. auf geringe Bandbreite reagieren zu k�nnen, oder Roaming-Geb�hren klein zu halten. Zum Beispiel m�ssen Apps von Museen, Ausstellungen, Konzertbetreibern oder Zeitschriften diese Anforderungen erf�llen

\section{Aufgabenstellung}
\begin{itemize}
\item Recherchieren, welche Produkte zur L�sung des Problems bereits bestehen und allf�llige Vor- und Nachteile dieser aufzeigen.
\item Aufzeigen welche L�sungsvarianten m�glich sind.
\item Einarbeiten in die Handheld Programmierung.
\item Auswahl von Techniken f�r die eigene Implementation.
\end{itemize}

\section{Zielsetzung}
Das Ziel der Seminararbeit ist es aufzuzeigen, welche Systeme, Prozesse und Design-Patterns wichtig sind und in Frage kommen um den oben genannten Anforderungen gerecht zu werden. Insbesondere die Schwierigkeit, m�glichst effizient aktuelle Daten dem App-Benutzer zur Verf�gung zu stellen soll dabei beachtet werden.

Falls m�glich k�nnen bereits gewisse Teile praktisch umgesetzt werden: z.B. kann ein CMS bereitgestellt werden, �ber welches der Betreiber der App die Inhalte bereitstellen und zur Aktualisierung f�r die App freigeben kann. Weiter kann eine App entwickelt werden, welche ihre Inhalte �ber das CMS bezieht und die definierten Prozesse umsetzt.


\section{Motivation}
Eine eigene Mobile App zu haben, ist heute f�r eine Firma oder eine Organisation fast schon so selbstverst�ndlich wie eine Website zu besitzen. Durch diese rasante Verbreitung von Mobile Apps wird es immer wichtiger, dass diese auch einfach aktuell gehalten werden k�nnen. Uns hat die Idee dieser Semesterarbeit deshalb fasziniert. 

Unabh�ngig von dieser Arbeit haben wir auch Anfragen aus dem Freundes- und Bekanntenkreis erhalten, welche sich f�r eine App interessierten, die f�r Promotionszwecke genutzt werden kann. Sie wollten wissen mit wie viel Aufwand eine solche App verbunden w�re. Da wir selbst keine Erfahrung in diesem Bereich hatten konnten wir auch keine Informationen geben. Dies hat uns aber zus�tzlich motiviert, nach geeigneten L�sungen zu suchen um die Erstellung von Apps, und vor allem die Erneuerung der Inhalte zu verinfachen.



