\chapter{Projektmanagement}

\section{Projektteam}
Das Projektteam besteht aus Alex Gustafson und Ramon Schilling. Die Recherchearbeit haben wir zwischen beiden Projektmitgliedern aufgeteilt und die Resultat jeweils miteinander besprochen. Die Implementation wurde haupts�chlich von Alex Gustafson gemacht, w�hrend die Dokumentation von Ramon Schilling erstellt wurde.

\section{Termine}
\begin{tabbing}
12. Sept. 2012	\= -	\= Kick Off Meeting	\\
26. Sept. 2012	\> -	\> Abgabe der Aufgabe \\
28. Nov. 2012	\> -	\> Abgabe Teaser	\\
05. Dez. 2012	\> -	\> Arbeitstreffen	\\
13. Feb. 2013	\> -	\> Abgabe Schriftliche Arbeit	\\
20. Feb. 2013	\> -	\> Pr�sentation	\\
\end{tabbing}

\section{Projektplanung}

\section{Aufwand}
Der Aufwand f�r die Seminararbeit sollte pro Person 75 h betragen. Da wir die Arbeit zu zweit machen, haben wir unsere Planung deshalb auf 150 h ausgelegt. \\

\begin{tabular}{|l|c|c|}
\hline 
\textbf{Beschreibung} & \textbf{Soll}& \textbf{Ist} \\ 
\hline 
Recherche von m�glichen Synchronisationsabl�ufen & 30 h & 30 h \\ 
\hline 
Recherche bereits erh�lticher L�sungen & 10 h & 9 h \\ 
\hline 
Testen verschiedener Synchronisationsabl�ufe & 8 h & 12 h \\ 
\hline 
Einrichten der Entwicklungsumgebung & 4 h & 5 h \\ 
\hline
Programmieren / Testen CMS & 35 h & 40 h \\ 
\hline 
Programmieren iPhone App & 42 h & 38 h \\
\hline
Beispiel App erstellen & 12 h & 10 h \\
%Beispiel App Definition Thema, Umfang & 2 h & 2 h \\
\hline
%Beispiel App Templates definieren, erstellen & 4 h & 4 h \\ 
%\hline 
%Beispiel App Daten erstellen, eintragen & 4 h & 4 h \\
%\hline
Dokumentation & 13 h & 14 h \\ 
\hline 
\textbf{Total} & \textbf{154 h} & \textbf{158 h} \\
\hline
\end{tabular} 


\section{Hilfsmittel}

\subsection{Versionskontrolle}
Um eine Versionskontrolle zu haben, und damit wir beide gleichzeitig am Projekt arbeiten konnten, haben wir bei Github \cite{Github} ein Repository erstellt und s�mtliche f�r das Projekt n�tigen Dateien dort eingecheckt.

\subsection{Dokumentation}
Dokumentation haben wir mit \LaTeX{} erstellt.

\subsection{Programmierung}
Als Entwicklungsumgebung f�r die Programmierung mit PHP \cite{php} hat uns PHP Storm \cite{PhpStorm} gedient.
\newline
\\F�r die iPhone App Programmierung haben wir Xcode 4 \cite{Xcode} genutzt.
\newline
\\Als Datenbank haben wir auf SQLite \cite{sqlite} zur�ckgegriffen.
\newline
\\Das Optimus Dashboard \cite{optimus} nutzen wir bei der Entwicklung des CMS.
\newline
\\F�r den File Upload nutzten wir elfinder \cite{elfinder}.